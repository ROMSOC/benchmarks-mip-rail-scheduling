In this section, the requirements for the scripts to run will be given. We will also provide a step-by-step manual for the usage of the benchmarks introduced.

\subsection{Requirements}
Our code is written in Python 3.7. Hence, the host machine needs to have a distribution of Python 3.7 or newer installed. Apart from packages available in the Python Standard Library, we also used \texttt{numpy} (for some numeric manipulations) and \texttt{networkx} (for the representation of graph objects and graph-related algorithms).

Our model was built with \texttt{gurobipy}, which is Python's API to the routines of the Gurobi solver. We have used Gurobi 9.1 in our work. Although Gurobi is a proprietary software, a free non-commercial license is available to all the members of academic community. 

\subsection{Usage of the benchmarks}
\paragraph{Step 0: Installation} Provided a Python interpreter and the required packages are available on your machine, you can simply clone from the ROMSOC Github repository: 

\noindent
\texttt{git clone https://github.com/ROMSOC/benchmarks-mip-rail-scheduling}

\paragraph{Step 1: Choice of instance} Navigate to \texttt{instances} directory. From there, navigate to a directory whose name matches the instance you would like to consider.

\paragraph{Step 2: Run model construction} There are two keywords which need to be supplied when running the scripts:
\begin{itemize}
    \item \texttt{\{weekly,monthly\}} -- determines the scope of the model. 
    If you chose \texttt{1M} as your instance, use \texttt{monthly},  otherwise use \texttt{weekly}.
    \item \texttt{\{write,nowrite\}} -- determines whether or not the resulting model will be written to an output file \texttt{model.lp}. 
\end{itemize}
For example, if you wish to consider the instance \texttt{1W\_1} and to generate an output file, you need to type:

\noindent
\texttt{python main.py weekly write}

\noindent
If you wish to consider the instance \texttt{1M} and \textbf{not} to generate an output file, you need to type:

\noindent
\texttt{python main.py monthly nowrite}



